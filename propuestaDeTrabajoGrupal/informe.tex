\documentclass[a4paper,11pt]{article}
\usepackage[utf8]{inputenc}
\usepackage{geometry}
\usepackage{multicol}
\usepackage{graphicx}
\usepackage{sectsty}
\usepackage{hyperref}
\usepackage{titlesec}

\geometry{top=1cm, bottom=1cm, left=1cm, right=1cm}

\sectionfont{\fontsize{11pt}{13pt}\selectfont\bfseries}

\titlespacing{\section}{1pt}{\baselineskip}{\baselineskip}


\title{}
\author{}
\date{}

\setlength{\topsep}{0pt}
\renewcommand{\refname}{}

\begin{document}

	\begin{center}
		\textbf{\fontsize{15pt}{17pt}\selectfont PROTOTIPO DE ALCOHOLÍMETRO APLICADO A VEHÍCULOS}
		
		Nahuel Valentín Pereyra () - Marcos Raúl Gatica (402006) - José Prado ()
		
	\end{center}
	
	\begin{multicols}{2}
		
		\section{MOTIVACIÓN}
			\begin{quotation}
				Se entiende por "alcoholemia" a la presencia de alcohol en sangre debido al consumo de bebidas alcohólicas, tales como los vinos, cervezas, etc. [1]. 
				
				Consumir alguna bebida con alcohol va disminuyendo gradualmente la capacidad motora y la visión del sujeto, llegando un punto donde el discernimiento se ve afectado y los reflejos, atención y agudeza visual disminuyen. Sumado a esto, la persona puede desarrollar una sensación de exaltación, dar falsa seguridad (como excederse de velocidad al conducir y olvidarse o ignorar protocolos de seguridad como el uso del cinturón) y tener dificultades en su inhibición. [2]
				
				La alcoholemia es medible en gramos/litro de sangre. Se puede medir directamente haciendo una extracción de sangre del sujeto o indirectamente por el aire expirado y siguiendo una fórmula (el trabajo propuesto por este documento usará la manera indirecta de esta variable)
			\end{quotation}
		
		
		\section{DEFINICIÓN DEL PROBLEMA Y SOLUCIÓN PROPUESTA}
		
		\section{REFERENCIAS}
			\begin{thebibliography}{9}
				
				\bibitem{leyDeTransito} 
				''Tránsito y consumo de alcohol, ley N° Ley 24.449'' disponible en: \url{https://www.argentina.gob.ar/justicia/derechofacil/leysimple/transito-y-consumo-de-alcohol#:~:text=%C2%BFQu%C3%A9%20es%20la%20alcoholemia%3F,el%20consumo%20de%20bebidas%20alcoh%C3%B3licas.}
				
				\bibitem{efectosConsumoDeAlcohol}
				''El consumo de alcohol y la seguridad vial'' disponible en: \url{https://www.argentina.gob.ar/seguridadvial/observatoriovialnacional/el-consumo-de-alcohol-y-la-seguridad-vial#:~:text=El%20consumo%20de%20alcohol%20disminuye,falsa%20seguridad%20y%20p%C3%A9rdida%20de}
				
				\bibitem{alcolemiaVariable}
				''El alcohol y su metabolización'' informe 
				
			\end{thebibliography}
		
	\end{multicols}
	
\end{document}
